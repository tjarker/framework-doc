\documentclass[12pt]{report}

\usepackage{comment}
\usepackage{graphicx}
\usepackage{hyperref}
\usepackage{xcolor}

\usepackage{listings}
\lstset{
  language=Scala,         % Language of the code
  basicstyle=\ttfamily\small,  % Font style and size
  keywordstyle=\color{blue},   % Keyword color
  commentstyle=\color{green},  % Comment color
  stringstyle=\color{red},     % String color
  numbers=left,                % Line numbers on the left
  numberstyle=\tiny\color{gray}, % Line number style
  frame=single,                % Frame around the code
  breaklines=true              % Line breaking
}

\usepackage[backend=biber,style=ieee]{biblatex}
\addbibresource{books.bib}
\addbibresource{papers.bib}
\addbibresource{pages.bib}

\author{Tjark Petersen}
\title{Thesis}

\newcommand{\name}{MyVerificationFramework}

\newcommand{\ttt}{\texttt}

\newcommand{\todo}[1]{\textcolor{red}{TODO: #1}}

\begin{document}
\maketitle

\section*{Abstract} %==========================================================================================================

\section*{Acknowledgements} %==================================================================================================

\section*{AI Declaration} %====================================================================================================

\newpage

\tableofcontents

\chapter{Introduction} %///////////////////////////////////////////////////////////////////////////////////////////////////////

\begin{lstlisting}
@main def helloWorld : Unit =
    println("Hello, world!")
\end{lstlisting}
- present topic
- present problem statement
- why is this relevant
- this is an explorative and problemsolving thesis
- explain the goal/objectives
- what methods will be used
- how is the thesis structured

- verification has a hard time keeping up with increasing design complexity
- verification is a bottleneck in the product development process (40-50\% of design cost) \cite{mehta2018asic}
- UVM is the most used verification framework
- UVM is overly complex and only a subset is actually used \cite{sutherland2015uvm}
- time to reevaluate verification tools

- 70\% of design effort is used on verification \cite[Ch. 1]{bergeron2012writing}
- testbenches manifest up to 80\% of code base \cite[Ch. 1]{bergeron2012writing}

- verification is important to make ASIC design a first time success \cite[Ch. 3]{bergeron2012writing}

\chapter{Background \& Related Work} %/////////////////////////////////////////////////////////////////////////////////////////

- what is the purpose of verification
- what tools do we have to ease verification
- what tools do we have to measure verification progress
- don't introduce "things" but introduce the problems/challenges they're trying to solve

- in this section a complete picture of what verification includes should be drawn in order to put the foundation for
understanding what a verification framework should provide

- to understand current state of verification, we need to look at general approaches, the langauges and their
features, frameworks, as well as methodologies

\section{Verification} %=======================================================================================================

Verifying a hardware design is about making sure that everything works as expected. \citeauthor{bergeron2012writing}
provides his definition as: \textit{"Verification is a process used to demonstrate the functional correctness of a
design"} \cite[Ch. 1]{bergeron2012writing}.

On the surface the task seems well-defined and straight-forward. However, looking closer, a series of challenges
become apparent.The specification, unless captured in a formal language, is open to interpretation. A
misunderstanding could lead to a slightly wrong implementation. The likelihood of catching this can be increased by
letting a different engineers implement and test the implementation, but it can't be reduced to zero \cite[Ch.
1]{bergeron2012writing}.

Unless a formal language is used for the specification, it is also impossible to \textit{prove} that an
implementation matches the specification. This is why \citeauthor{bergeron2012writing} uses the word
\textit{demonstrate} instead. Verification is thus question of confidence and about convincing oneself that the
implementation matches the intent captured in the specification. This confidence is built by demonstrating that the
implementation behaves as the specification prescribes in a selection of scenarios \cite[Ch. 1]{bergeron2012writing}.

These scenarios are captured in a verification plan. It holds a list of features and a set of conditions under which
a certain behavior should be observed. In short, it defines a series of test cases for each feature, which by itself
should demonstrate that the feature works correctly. Whether all edge-cases have been captured in these
demonstrations is another question, which the verification team has to convince themselves of through a thorough
process \cite[Ch. 1]{bergeron2012writing}.

The fact that correctness can't be proven based on a non-formal specification can't be helped by using model checking
methods either, which can prove that a certain property holds for a given implementation. The properties themselves
are an interpretation of the specification and thus subject to the same ambiguities and misunderstandings \cite[Ch.
1]{bergeron2012writing}.

The tool by which correct behavior of a feature is demonstrated is the \textit{testbench} in a process referred to as
\textit{functional verification}. Functional verification focuses on design intent and that the implementation
provides certain functionality. The testbench is a closed system which tries to emulate the environment of the design
in a controlled manner. It exercises the design such that it becomes evident, by manual or automatic inspection, that
a feature is working as intended \cite[Ch. 1]{bergeron2012writing}.

\begin{comment}

\cite[Ch. 1]{bergeron2012writing}
- important question is: what is being verified?
- the answer lies in the specification
- how precise is the specification, are there ambiguities?
- how complete is the specification, are there corner cases not covered?
- unless the spec is captured in a formal languages, it is impossible to prove that an implementation matches the spec
- verification is thus a process of convincing oneself that the implementation is correct beyond a reasonable doubt

- the spec itself is not used though, it is an interpretation of the spec by an engineer which is used to design an implementation but also the testbench
- important for these two processes to be performed by different people, such that there is at least a chance of discovering misinterpretations

- the problem is inherent in the transformation of a specification
- also not fixed by model checking methods, where properties are still DERIVED from spec

\cite[Ch. 1]{bergeron2005verification}
- progress is measured in number of features demonstrated to work correctly

\cite[Ch. 1]{bergeron2012writing}
- functional verification focuses on demonstrating that a certain functionality meets the intent captured in the specification
- in order to do this, the design is exercised in a controlled environment, the testbench
- it exercises the design such that it becomes evident, by manual or automatic inspection, that a feature is working as intended
- testbench itself is a close system which tries to emulate the environment of the design in a controlled way

\cite[Ch. 1]{bergeron2005writing}
- functional testing can be differentiated by the degree on which signals internal to the DUV are used to determine the correctness of the design
- problem lies in controllability and observability of the DUV
- how easy is it to trigger certain behavior, how easy is to it to see that the behavior was executed correctly
- black-box testing only uses the inputs and outputs of the DUV, can especially suffer from low observability, since an potential error has to be detected far away from the actual cause
- other extreme is white-box testing where full visibility of the internals is used in the test, it suffers from its lack of generality -> a change in the DUV requires a change in the testbench
- a compromise between both can be referred to as grey-box testing, tries to limit dependency on implementation specific stuff while still providing good observability and controllability

\end{comment}

\subsection{Functional Verification} %-----------------------------------------------------------------------------------

% also talk about assertions

Functional verification can be differentiated by the degree on which signals internal to the device under
verification (DUV) are used to determine the correctness of the design. From a maintainability perspective, it makes
most sense to defer from accessing any internal signals in the implementation as part of the testbench and seeing the
DUV as a black-box with a certain interface. Changes in the implementation would in this case require no changes in
the testbench. It also means that different implementation attempts sharing the same interface could reuse the same
testbench. However, this approach can suffer from a low controllability and observability. Controllability refers to
how easily a certain functionality can be triggered inside the DUV by driving its inputs. Observability on the other
hand, refers to how easily the effect of a certain functionality being triggered becomes visible at the outputs of
the DUV \cite[Ch. 1]{bergeron2012writing}.

In complex designs some features may suffer from low controllability and/or observability. In this case, white-box
testing can be used, which provides full access to the internals of the DUV. Register state as well as outputs of
functional units related to the functionality under test can be directly accessed to verify its correctness. This
approach suffers from the fact that each change in the implementation may require a change in the testbench. A
compromise between black-box and white-box testing is grey-box testing. It tries to balance the dependency on
implementation details with maximizing observability and controllability \cite[Ch. 1]{bergeron2012writing}.

Functional verification using testbenches and test cases can be aided by assertions, a way to check if a property
holds in the implementation. Assertions can for instance be used to check if a certain condition holds at a certain
point in time, over a certain period of time or when another condition holds. They are placed directly in the source
code next to the functionality that they are checking a property of. This gives them maximum visibility leading to
maximum observability. These assertions can be used alongside a normal testbench which applies stimulus to the DUV
though its interfaces. Should a property specified in an assertion at any point not hold, the error will point
directly to the source of the problem. This is in contrast to a traditional testbench where an error occurring on the
interface has to be traced back to the actual cause \cite[Ch. 14]{mehta2021introduction}.

Up until now, only dynamic functional verification has been discussed, where the behavior of the implemented design
is observed by \textit{simulating} it. This is in contrast to static functional or \textit{formal} verification,
where a tool will try to disprove the property by analyzing the logic of the design itself. If the tool is able to
disprove the property, it will also have a counter-example which shows how the property can be violated. If the tools
is not able to disprove the property, it is guaranteed to hold under any condition which the design can be exposed to
\cite[Ch. 14]{mehta2021introduction}.

This can obviously be a powerful tool for finding bugs in an implementation. The limitations however are twofold.
Properties are effectively small formal models of the functionality under test and are hard to develop. Also, the
performance of analyzing a property deteriorates quickly with the complexity of the design. The process suffers from
the so-called "state space explosion" problem. The tool has to analyze how applying a certain set of inputs will
develop cause future state transitions. In a complex design, the number of possible choices which a system has,
leading to new choices and so on, just becomes to large to handle within reasonable time. As a solution, hybrid
approaches exist, where static methods are combined with dynamic ones. The design is lead to a certain state in a
simulation and then it is statically checked whether the property can be violated from this state \cite[Ch.
14]{mehta2021introduction}.

\subsection{Constrained-Random Verification} %---------------------------------------------------------------------------------

With ever increasing design complexity, it is not feasible to write directed test cases to methodically exercise and
verify each feature under each possible condition as outlined in the verification plan. Instead, randomized stimulus
can provide a way to increase productivity. If a long enough steam of random stimulus is applied to the interfaces of
the DUV, it is likely that many of the test cases outlined in the verification plan will occur naturally, without any
additional effort. It is even possible that a random stream of stimulus may create unforeseen conditions, improving
the thoroughness of the verification effort \cite[Ch. 1]{bergeron2005verification}.

Of course, it does not make sense to apply completely random stimulus to the DUV interfaces. Only a fraction of the
applied stimulus would often be valid for complex interfaces like ethernet or PCIe ports. A mechanism to define what
a valid transaction for a given interface looks like is needed. This is achieved by constraining the random stimulus.
Constraints limit the set of legal assignments to the set of random variables used to derive a random stimulus for a
given interface \cite[Ch. 3]{bergeron2012writing}. These constraints could for instance capture that the length field
of an ethernet frame has to match the actual length of the payload, but also that the address in a bus transaction
has to be aligned to the size of the transfer.

The verification runtime will, given the constraints on the random variables, generate a set of assignments which
satisfies all constraints using a constraint satisfaction problem (CSP) solver. In addition to inter-variable
constraints, distribution constraints are highly relevant,
in order to ensure that the random stimulus is representative of the actual behavior of the DUV \cite[Sec. 7.5]{flake2020a}.

The verification approach of using random stimulus alongside constraints is called constrained-random verification
(CRV). It promises an increase in productivity, which is given by a single CRV testbench being able to exercise
potentially many of the test cases outlined in the verification plan. This allows for an approach where verification
is started off using CRV, and only then more specific tests are developed to target functionality which is difficult
to exercise using CRV \cite[Ch. 3]{bergeron2012writing}. The question of couse is, how to measure what has been
tested and what has not.

\begin{comment}

start of with CRV and then add specific tests with constrains for uncovered features \cite[Ch. 3]{bergeron2012writing}

by exercising many of
these features with a single, very long stream of random stimulus . If the stimulus
generator is properly constrained to produce valid transaction patterns, it may not only produce the test cases
captured in the verification plan, but may even lead to conditions not considered in the verification plan, like some
hard-to-trigger edge cases \cite[Ch. 3]{bergeron2012writing}.

\cite[Ch. 13]{mehta2021introduction}

- idea is quite simple
- but achieving actually exhaustive random stimulus can be difficult

\end{comment}

\subsection{Coverage} %--------------------------------------------------------------------------------------------------------

A testbench is measuring how well the DUV lives up to the specification. But how well is the collection of
testbenches living up to checking all aspects outlined in the specification? In a case, where only directed tests are
used, the answer is simple since each test case can be traced back to a specific feature in the specification. When
CRV is used on the other hand, it becomes essential to have some kind of tool which can actually measure which
functionality has been exercised and which has not \cite[Ch. 15]{mehta2021introduction}.

One tool which can give an idea about how thoroughly a design has been tested is code coverage. It provides insight
into how much of the source code, structurally but also semantically, has been exercised by a test suite. By adding
instrumentation to the design, the activity of certain aspects of the design can be monitored while running a
testbench \cite[Ch. 2]{bergeron2012writing}.

At the most basic level, line coverage checks whether a line of code has been executed. More useful is statement
coverage which considers each individual statement independent of its position in the code. A statement could for
instance be an assignment operation. Branch coverage concerns itself with control flow such as if-statements,
measuring how many of the different paths through a piece of code have been taken. Finally, finite-state machine
(FSM) coverage recognizes FSM patterns in the source code. It can report which states of the FSM have been visited
and which transitions have occurred. FSM coverage doesn't know about valid transitions though and as such all
possible transitions are reported. Illegal ones have to be filtered out manually to extract a useful metric out of
FSM coverage data \cite[Ch. 15]{mehta2021introduction}.

Code coverage doesn't however know anything about the functionality of the design. A low code coverage indicates that
some functionality of the design has not been exercised and thus tested. A 100\% code coverage on the other hand does
not mean that the verification task is done \cite[Ch. 2]{bergeron2012writing}.

To measure the coverage of a test suite at the functional level, a more general tool is needed. This so-called
functional coverage is manifestation of the verification plan and thus of the design specification, tracking whether
each functionality has been exercised under all relevant conditions \cite[Sec. 7.6]{flake2020a}. It can't be
automatically derived from a non-formal specification and thus has to be manually defined \cite[Ch. 15]{mehta2021introduction}.

The idea behind functional coverage is to monitor state, transitions, changes to variables or expressions and a
combination of these (so-called cross-coverage). For each functionality, different cases of interest have to be
defined and declared as so-called bins. Signals or variables can have different bins for specific values or ranges
which may be augmented with predicates. At a given sampling event these bins record whether they have been hit, i.e.
whether the condition they represent has been observed. The goal is to have all bins hit at least once, indicating
that all aspects of the design have been exercised \cite[Sec. 7.6]{flake2020a}.

\todo{maybe give more concrete example, especially for cross coverage}

\begin{comment}

\cite[Ch. 15]{mehta2021introduction}
- how to measure how well the testbench is performing?
- especially in the case of CRV this becomes essential, since it is not clear what functionality has been tested under which conditions

\cite[Ch. 2]{bergeron2012writing}
- code coverage provides insight into how much of the source code or structure of the DUV has been exercised by a test suite
- via adding instrumentation to the design, the activity of certain aspects of the design is monitored

\cite[Ch. 15]{mehta2021introduction}
- at the simplest level, line coverage checks whether a line of code has been executed
- more useful is statement coverage which looks at each statement independent of its position in the code
- branch coverage concerns itself with control flow, it measures how many of the different paths through a piece of code have been taken
- finally, FSM coverage recognizes FSM patterns in the source code
- it can report which states have been visited, which transitions have occurred
- does not know about valid transitions though, illegal ones have to be filtered out manually

- code coverage can indicate that the verification is NOT down, never whether it is done \cite[Ch. 2]{bergeron2012writing}

- a more general concept of coverage is needed, functional coverage \cite[Sec. 7]{flake2020a}
- it has to be based on the design specification, can't be derived and thus manual work \cite[Ch. 15]{mehta2021introduction}

\cite[Sec. 7]{flake2020a}
- essentially about monitoring state, state transitions, changes to variables and expressions and combinations of these (cross)
- one bin for each state, transition, cross... which corresponds to a functional aspect of the DUT
- all bins should have hits

\end{comment}

\subsection{Testbench Abstractions \& Reusability}
%------------------------------------------------------------------------------------------

It is clear that it is not sustainable to write monolithic testbenches to demonstrate the correctness of each feature
in a design. The testbenches would become too complex and large to maintain, and a lot of development time would be
used to code some of the foundations each testbench relies on. A testbench can be split into two parts at the highest
level: the part which eases the interaction with the DUV programmatically, the so-called test harness, and the part
which is specific to the test case \cite[Ch. 6]{bergeron2012writing}.

If these two parts were to be split such that the test harness could be reused across all testbenches for a given
DUV, the question of what interface the test harness exposes to the test case code arises. Since no abstraction is
achieved by working with the ports of the DUV directly, a higher level of abstraction is needed. This is where the
concept of transactions comes in. A transaction is an operation on an interface, as abstract as the transmission of a
TCP package but also low level like an AXI bus write operation \cite[Ch. 1]{bergeron2005verification}.

Given the abstraction level of transactions, components are needed to bridge the gap between the transaction level
and the pin level of the DUV. These components are called transactors. Since the gap between the highest level of
abstraction and actual pin interactions can be quite large, such as in the case of TCP packages, intermediate levels
of abstraction should be introduced, with transactors bridging the gap between them. This means that transactors can
be layered and composed to meet the needs of a specific test case \cite[Ch. 4]{bergeron2012writing}.

The lowest levels of transactors, which interfaces the pins of the DUV, are called bus-functional models (BFM). In
contrast to other transactors, BFMs are concerned with timing \cite[Ch. 4]{bergeron2012writing}. They encapsulate the
protocols of an interface and translate a transaction into a potentially multi-clock cycle interaction with the
interface pins of the DUV \cite[Ch. 3]{salemi2013uvm}.

Transactors assume that the DUV correctly handles all interactions of the abstraction levels below it. As such, a
test should exist, verifying that each level of abstraction is correctly handled by the DUV. For instance, in a
packet based system with multi-cycle transmission, one test should verify that parts of a packet are correctly sent,
such that all other test can safely assume this property \cite[Ch. 6]{bergeron2012writing}. This methodology centered
around transactions and transactors allows for the structuring of a testbench into reusable components, which can be
composed to meet the needs of a specific test case and thus can increase the test case development productivity.

\begin{comment}

\cite[Ch. 1]{bergeron2005verification}
- transaction is an operation on an interface, as abstract as the transmission of a TCP package but also low level like an AXI bus write operation

\cite[Ch. 6]{bergeron2012writing}
- tb has two parts test harness and test case specific code
- harness is possible to reuse

- it is clear that

\cite[Ch. 3]{salemi2013uvm}
- BFM concerned with signal level -> first step towards transactions
- BFM encapsulates the protocols of an interface
- instead of driving signals, call methods which advance sim time while driving pins
- in SV this is implemented using the interface construct

\cite[Ch. 6]{bergeron2012writing}
- BFMs can be layered, where a higher level BFM calls a lower level BFM
- each BFM assumes correctness of some interaction with the DUT
- this can be demonstrated in one separate test such that all others can rely on this

\end{comment}

\section{Verification Languages \& frameworks} %===============================================================================

\todo{meta text}

\subsection{SystemVerilog's Predecessors} %------------------------------------------------------------------------------------

In the mid 1990s, some of Verilog's limitations became apparent as the complexity of designs grew significantly. Next
to movements trying to replace HDLs altogether with general-purpose programming languages through high-level
synthesis, others tried to strengthen Verilog by making it more capable especially in terms of its specification and
verification features \cite[Sec. 6]{flake2020a}.

One such attempt was Superlog. A team at Co-Design Automation Inc. thought that Verilog lacked features of
general-purpose languages to be efficient for verification. Being focused preserving the high performance of Verilog
simulators, the group turned to C for inspiration. The resulting Superlog language added C data types like enums and
structs, dynamic memory allocation, but also dynamic data types likes strings, dynamic arrays and queues. Another
reason to draw inspiration from C was to make C code and Superlog easily interoperable which was useful for
integrating C models in verification code \cite[Sec. 6]{flake2020a}.

Another addition over Verilog was the introduction of interfaces, which on the one hand allow bundling of signals
with directionality, and on the other hand allow for exposing functions from one module to another. The latter
feature was deemed useful for transaction level models, allowing Superlog to be used as a specification language
\cite[Sec. 6]{flake2020a}.

Another direction of development focused only on the verification aspects and lead to the development of hardware
verification languages (HVL). These were languages specifically tailored to the task of verifying hardware designs,
adjusting the level of abstraction and the expressiveness of the language to needs of the verification tasks. One of
these languages was Vera, initially developed at Sun Microsystems in 1994 \cite[Sec. 7]{flake2020a}.

The Vera language was built around a co-simulation engine which could interface different hardware simulators, and
was meant to be HDL agnostic. The level of abstraction in interacting with the DUT is slightly lifted, by removing
the need for explicit timing and focusing instead on clock cycle based interaction. The IO of the DUT is captured in
interfaces which are associated with a clock, outputs defining a skew relative to a clock edge when they should be
driven and inputs defining a skew relative to a clock edge where they should be sampled. These interfaces can be
bound to ports which in turn can be passed around the testbench to provide access to the DUT. Signals can be directly
referenced and assigned in the code, with the runtime taking care of the actual timing in the simulation running in
the background \cite[Sec. 7]{flake2020a}.

Vera offers flexibility in scheduling interactions with the DUT through the @ operator, which allows to defer
assignments and assertions to future clock cycles or synchronize the current thread with a signal change. Assertion
have a timing dimension, e.g. allowing to specify intervals in which the condition should become true or continuously
hold . A Vera program starts in a single thread, from which it can spawn new threads using a fork-join model.
Multi-threading constructs such as synchronization events, semaphores and mailboxes are part of the language
\cite[Sec. 7]{flake2020a}.

One of the maybe most important features added by Vera is the inclusion of object-oriented programming (OOP). Vera
introduces classes and inheritance, which allows for a more structured and modular testbench design. Equally as
important are the addition of constrained randomization and functional coverage collection. Class fields in very can
be marked with the rand keyword. A call to the \ttt{randomize} method on an object of the class will then randomize
all fields marked with rand according to the constraints specified in the class. Functional coverage can be collected
by defining bins for states, transitions and general expressions or the cross-product of these. A later addition to
the Vera language when it was adopted as an open standard by Accellera as OpenVera, was the introduction of Open Vera
Assertions (OVA). These rely on linear temporal logic, a type of formal logic used to describe sequences of events
over time, which allows to capture highly complex behavior in assertions \cite[Sec. 7]{flake2020a}.

\begin{comment}

- mid 1990s Verilogs limitations became apparent as design complexity grew
- discussed replacing HDLs altogether with C++ or java for hardware design
- movement to strengthen Verilog lang -> Superlog
- verilog lacked features of general purpose programming language to be efficient for verification
- Superlog mostly superset of verilog with inspiration from C for data types like enums and structs and primitives ike strings and arrays, queues, dynamic memory allocation
- easy to interface with C code
- introduces interface construct to bundle signals but also allow exposing functions for transaction level modelling by associating them with an interface, they can operate on the signals of the interface

- other developments focussed only on the verification aspects
- verilog not designed for modular complex testbenches
- not very concise in expressing verification intent
- lead to development of HVLs

- one of these was Vera
- a cosimulation engine interfacing with hardware simulators
- language agnostic
- lifts abstraction level by focussing on steady state behavior
- for this signals are associated with clock and samples relative to that clock
- skews relative to clock edges are provided to define sampling and drive times
- IO of dut is captured in interface which can be bound to ports which can be freely passed around the testbench to provide access to the DUT
- signals can be directly referenced and assigned in the code, the runtime takes care of the actual timing
- interactions can be deferred to future clock cycles using the @ operator
- assertions can be directly written as boolean expressions with a potential interval or point in time when the condition should hold
- synchronization apart from the clock cycle granularity is provided by @(signal) or @(posedge signal)
- program starts from single thread and can spawn new threads using fork-join model
- multi-threading constructs like events for synchronization, mailboxes and semaphores are part of the language
- adds object oriented features like classes and inheritance
- adds crv features by marking class fields with rand keyword and specifying constraints
- finally vera adds functional coverage by allowing to define coervage bins for states, transitions and general expressions
- a later addition to vera was OVA open vera assertions, which relied on LTL to specify complex temporal properties

OpenVera \cite[Sec. 7, pp. 51-??]{flake2020a}

\end{comment}

\subsection{SystemVerilog} %---------------------------------------------------------------------------------------------------

Towards the early 2000s, motions where made to combine the progress which had been made in the development of more
modern hardware design and verification languages like Superlog and Vera into one standardized language. The idea was
to have one language for design and verification, enabling modern verification techniques such as CRV, functional
coverage collection and complex assertions. The result was SystemVerilog which was first publicized as a standard in
2002 by Accellera and after several revisions and rework was made a IEEE standard in 2005 \cite[Sec. 9]{flake2020a}.

SystemVerilog is mostly based on Superlog and Vera, the API of latter only being slightly changed to harmonize the
syntax with the rest of the language. The SystemVerilog subset for assertions wasn't only based on Vera's OVA, but
also drew inspirations from other specification languages such as the property specification language PSL, an
Accellera and later IEEE standard based on temporal logic. As such SystemVerilog is actually four languages in one
\cite[Ch. 1]{mehta2021introduction}:

\begin{enumerate}
  \item A subset for synthesis
  \item A subset for object-oriented verification
  \item A subset for coverage collection
  \item A subset for assertions
\end{enumerate}

\subsection{UVM} %-------------------------------------------------------------------------------------------------------------

The verification features which SV offers are by themselves not enough to create reusable testbenches. They provide
the raw mechanisms for creating modern constrained-random, self-checking testbenches with coverage collection and
bus-functional models, but the language itself does not prescribe a way of organizing the different responsibilities
in a testbench such that large testbenches become manageable and such that parts of a testbench can be reused in
another project or be bought from another company. To achieve this, design principles and a methodology are needed for
the construction of testbenches \cite[Sec. 9.2]{flake2020a}.

Acknowledging the problem, each EDA vendor had developed their own reuse methodology: Mentor Graphics had the Open
Verification Methodology (OVM), Cadence had the Universal Reuse Methodology (URM) and Synopsys had the Verification
Methodology Manual (VMM). Due to the fear of vendor lock, none of these methodologies gained widespread adoption
\cite[ch. 4.1]{mehta2018asic}. Under the umbrella of the Accellera Systems Initiative, a standards organization
supported by the EDA industry, a merger of the different methodologies was attempted. The result was the Universal
Verification Methodology (UVM) which is mainly based on the OVM \cite[ch. 4.1]{mehta2018asic}.

The UVM provides standardized ways to create testbench infrastructure which is hierarchical and where
responsibilities are divided over different components. Furthermore, it introduces the concept of phases,
synchronizing components throughout the different steps of a test case. Finally, the UVM raises the level of
abstraction of the testbench by working with transactions which encapsulate a potential multi-clock cycle interaction
with the DUT in one object \cite[ch. 4.1]{mehta2018asic}. In the rest of this section, the core concepts of the UVM
will be introduced. The focus will be primarily on the functionality and ideas. Implementation details will only be
highlighted when they are relevant to the discussion.

\subsubsection{UVM Testbench Structure} %--------------------------------------------------------------------------------------

The UVM offers a separation of concern between three different perspectives: the test writer, the sequence writer and
the environment writer \cite{sutherland2015uvm}. The test writer constructs a test case to exercise a specific
feature by chaining a series of sequences together and applying them to the DUT by handing them to the environment.
He does not need to know how the sequences are constructed or how the DUT is driven. The sequence writer creates
sequences of transactions, so-called sequence items, which achieve a specific goal in the DUT. He does not need to
know how the transactions are
driven onto the DUT either. Only the environment writer concerns themselves with the translation of transactions into
pin-level signals and the driving of these signals onto the DUT. The environment writer also creates the analysis and
checking infrastructure of the testbench like coverage collectors or scoreboards. The environment manifests the
static infrastructure of the testbench which is shared between different test cases. It contains components which
direct the stimulus generated by the test case to the relevant DUT interfaces and transactions observed on the DUT
interfaces to analysis components for coverage
collection or checking against a model \cite{sutherland2015uvm}.

The full UVM testbench hierarchy is presented in \ref{fig:uvm_tb}. In a complex system, the top-level environment
itself consists of
multiple other environments, which are specific for sub-systems of the DUT. Inside each environment, there are a
series of agents which are specific to the interfaces which are part of the sub-system. Additionally, components
collecting coverage information at the sub-system level may be present. Finally, the environment contains a
scoreboard which compares the transactions observed on the DUT interfaces with transactions produced by some kind of
model, the predictor.

An agent is specific to one interface, for instance AXI4 or SPI, and bridges the gap between the transaction level
and the driving and reading of the pins of the DUT. Agents come in two forms: passive and active. Passive agents only
observe transactions on the interface pins and send them to analysis components such as scoreboards or coverage
collectors. Active agents also accept transactions which they drive onto the interface pins.

This compartmentalization of the testbench aims at promoting reuse where possible and should make maintaining large
and complex testbench systems easier. Furthermore, external IP blocks, for instance an AXI agent, should be easy to
integrate into a testbench.

\subsubsection{Communication between components} %----------------------------------------------------------------------------

The communication between the different components of the testbench is facilitated by the TLM (Transaction Level
Model) interface. The TLM is built around the idea of push and pull channels. The active side of a channel connection
(the producer in a push channel, the consumer in a pull channel) is called the \ttt{port} while the passive side is
called the \ttt{export}. The \ttt{export} provides a method for handling the receiving or sending of a transaction
which is triggered by the \ttt{port}. Most of the time, two components are connected with a buffer between them such
that a buffered channel is created. In this case the buffer provides the two passive channel ends such that the
sender can call \ttt{put} at any time and the receiver can call \ttt{get} at any time. Calls to these methods are
blocking when no data is in the buffer. In the testbench, ports can be forwarded from a child component to a parent
component, exposing the port at a higher hierarchy level. This is for instance the case in an active agent, where the
port for the transaction stream to the driver is exposed by the agent itself. Next to single-producer-single-consumer
channels, the TLM also provides one-to-many channels with so-called \ttt{analysis\_port}'s. These are used to
broadcast transactions to multiple analysis components like coverage collectors or scoreboards \cite[ch. 4.5]{mehta2018asic}.

\subsubsection{Phases}
Another concept introduced by the UVM are phases. Working with more elaborate DUTs with complex reset sequences, it
becomes natural to split the handling resetting the DUT from the actual test case. In a parallel system like UVM
where multiple components run concurrently, the challenge then is how the components should agree when the test case
starts. As another example, analysis components in the testbench need to know when the test case has finished in
order to generate summary statistics \cite[ch. 4.6]{mehta2018asic}.

UVM solves this synchronization issue by defining a series of phases with each component having the option run during
a phase by overriding a specific method. An overview of all UVM phases can be seen in Fig. \ref{fig:uvm_phases}. The
phases can be divided into three groups: build phases, run phases and cleanup phases. In the build phases, the
testbench hierarchy is set up. The \ttt{build} phase runs top-down through the hierarchy. Here all component
instances are constructed using the UVM factory and each component can configure its children by setting the
ConfigDB. The \ttt{connect} phase runs bottom-up and is used to connect the TLM channels between the components
\cite[ch. 4.6]{mehta2018asic}.

In the run phases, the actual test case is run and the DUT is stimulated. That means that these phases are the only
ones consuming simulation time, signified by the usage of \ttt{task}'s instead of \ttt{function}'s in SystemVerilog.
As it can be seen in Fig. \ref{fig:uvm_phases}, next to a series of fine-grained run phases, there is also the
\ttt{run} phase itself which spans the whole simulation. The most important run phases are \ttt{reset},
\ttt{configure}, \ttt{main} and \ttt{shutdown}. During the \ttt{reset} phase, the DUT can be forced into a known
state. The \ttt{configure} phase should be used to set up the DUT for the actual test case, e.g. by setting up
memories. The \ttt{main} phase is where the actual test is performed, i.e. the stimulus is applied to the DUT. The
\ttt{shutdown} phase can be used to wait for all effects of the applied stimulus to take place \cite[ch. 4.6]{mehta2018asic}.

As noted by the authors of \cite{dvcon2014reset}, reset behavior should not only be verified at the start of a test
but potentially also midtest in CRV test cases. The UVM does not directly prescribe a way to handle this, and altough
jumping between phases is possible, this poses the challenge of gracefully terminating the current phase in all
components. The authors propose a reset monitor which invokes cleanup hooks in all components when a reset is detected.

The cleanup phases in the UVM are used for analysis components to determine whether the test was passed successfully
and whether secondary goals such as coverage were met. The purpose of each phase in this group is very narrow and is
evident by the name \cite[ch. 4.6]{mehta2018asic}.

\subsubsection{UVM Sequence Items and Sequences} %-----------------------------------------------------------------------------

\cite[Ch. 4.3]{mehta2018asic} pp. 25-26

first talk about sequences after rough structure of UVM testbench is established

\subsubsection{Component Types} %----------------------------------------------------------------------------------------------

now dive deeper into what the individual components do

% driver
A driver has a port on which it can receive transactions. Normally, a driver enters an infinite loop in the \ttt{run}
phase where it wait for transactions and applies them either directly to the DUT or does so via a BFM. In order to
interact with the DUT, it is necessary in SystemVerilog to use a \textit{virtual interface} which effectively is a
handle to the pins of the DUT. The driver gets the next transaction through a call of \ttt{get\_next\_item} and
signals that the transaction has been applied by calling \ttt{item\_done}. The driver can provide feedback to the
code generating the sequence by passing a new transaction to the \ttt{item\_done} call. Special care has to be taken
when an interface is pipelined, i.e. when the DUT can accept multiple transactions before the first one is
acknowledged. In this case, the driver has to keep track of the transactions which are in flight and has to make sure
that the transactions are applied in the correct order \cite[ch. 4.7]{mehta2018asic}.

% sequencer
The transactions a driver receives could come from multiple sequences which are run at the same time. For this
purpose, the UVM introduces the sequencer which is located in an agent. The test case can hand of multiple sequences
to the sequencer which then decides how to interleave the transactions from the sequences depending on its
implementation. Usually \todo{finish}

\todo{virtual sequencer and sequences}
\cite[Ch. 23]{salemi2013uvm} p. 175

% monitor
A monitor holds, like the driver, a handle to the pins of the DUT. It observes the state of the pins and translates
the observed activity into transactions which are published via one or multiple analysis ports.

% subscriber
A series of subscribers can then receive these transactions and act upon them, for instance collecting coverage
information or comparing the transactions to a model. A coverage collector looks at the incoming transactions and
uses SystemVerilog functional coverage features to keep track of statistics such as whether a read was performed after a write.

% scoreboard
The task of verifying that the transactions produced by the DUT are correct is handled by the scoreboard in the UVM.
The UVM does not prescribe a way of how exactly to achieve this. Since correctness has to be checked for an arbitrary
stream of transactions, some form of a model is usually needed. Depending on the complexity, this model could be
included in the scoreboard or be its own component which just like the DUT receives one or multiple transaction
streams and produces another set of transaction streams. The scoreboard is in this case reduced to comparing the
result streams of the DUT and the model.

\subsubsection{Configurability} %----------------------------------------------------------------------------------------------
%- Configurability
%  - factory
%  - configDB

The introduction of the component system, each with their well defined responsibilities, is only one part of the way
towards reusable and scalable testbenches. Not only concepts and abstractions are necessary, but also guidelines and
programming patterns for how the testbench code should be implemented. How can a specialized component be derived
from another for a specific test case? How can different versions of a component be easily swapped out? How can a
component implementation be made configurable in a way that also works well with derived components?

The answer to the first question is inheritance, which is available through SystemVerilogs OOP features. A component
can be derived from another and override methods to specialize its behavior. The answer to the question of how to
swap out different versions easily which the designers of UVM and its predecessors chose, is the factory pattern. In
a factory pattern, objects aren't created through their constructor, but through a factory method which dynamically
determines which object to create, possibly based on a set of standardized parameters. In UVM, all UVM objects are
registered with a global factory. When a certain component A should be used instead of another component B, a factory
override can be issued, resulting in the creation of component A in all places where component B would normally have
been created. Of course A should be a subtype of B, such that A and B share the same interface. This allows for very
simples swapping out of components in a testbench without making the test code parameterizable with respect to the
different component implementations \cite[Ch. 13]{salemi2013uvm}.

Due to the global nature of the UVM factory, a one-siz-fits-all approach has to be adopted for what parameters may be
passed to the factory method. In the UVM, only the name for the object is given when an instance is produced:
\ttt{my\_class::type\_id::create("MyName")}. This raises the question of how to configure instances of UVM objects,
since it is impossible to pass parameters as it would usually be done in a constructor. The UVM solves this problem
with a configuration database, the so-called \ttt{uvm\_config\_db}, a global dictionary which is aware of the
component hierarchy in the testbench. One component can set a parameter in the configuration database and a
sub-component will be able to query the database for the parameter. In order to allow for a component to have
multiple sub-components of the same type with different configurations, the configuration database accepts a
hierarchical path when depositing a value in the dictionary. The deposited value will only be visible to the
referenced component and all its potential sub-components \cite{} \todo{what reference for configdb}.

Together, the factory and configuration database provide the means for a test case to tailor the static testbench
environment to its needs. The factory is responsible for swapping out full implementations, while the configuration
database makes it possible to change the parameters a specific component works with \cite[Ch. 4.3]{mehta2018asic}.

\begin{comment}

- components are one step towards a reusable and scalable verification environment, providing concepts and abstractions
- but reuse and scalability also have to be tackled at the programming level

- not alone good enough to provide primitives
- we also need to concern ourselves with software design patterns to create resuable testbenches
- how do we swap out different implementations? -> factory
- how do we specialize an implementation? -> inheritance
- how do we make an implementation configurable? -> configdb

\end{comment}

\subsubsection{UVM Tests} %----------------------------------------------------------------------------------------------------

\todo{RAL is actually just a transactor system with a reg model to generate meaningful transactions}

\todo{utilities: macros, reporting...}

\cite[Ch. 19]{salemi2013uvm}

\subsection{Open-Source Alternatives} %----------------------------------------------------------------------------------------

\todo{meta text}

\subsubsection{Cocotb} %-------------------------------------------------------------------------------------------------------

Cocotb is an open-source project under the FOSSi foundation, which focuses on providing a productive and
vendor-agnostic testbench framework for Verilog, VHDL and even mixed-signal designs \cite{cocotb}. Cocotb, short for
Coroutine-based Co-simulation Testbench, uses coroutines in python as an efficient way of having many simulation
threads cooperatively interface one simulation. Simulation events are modelled as asynchronous function calls which
can be "awaited", resulting in the waiting thread to suspend execution until the simulation backend triggers the event.

While taking advantage of the low learning-curve and rich library ecosystem of Python, Cocotb also suffers from the
drawbacks of python like the lack of static typing and the overhead of the Python interpreter. Due to its ease of use
and support for all major simulators, including commercial ones, Cocotb is a popular choice for a modern open-source
testbench framework. It does however not provide the infrastructure for building scalable and reusable testbenches
like the UVM does.

\subsubsection{Chiseltest} %---------------------------------------------------------------------------------------------------

Chiseltest was developed as part of the Chisel hardware construction language and is a Scala-based testbench
framework \cite{chiseltest}. Due to a change in the Chisel compiler infrastructure, it has been recently deprecated,
but a replacement is under development by the Chisel community. While meant to be used with Chisel designs, Verilog
designs can also be included via Chisel. The number of supported simulators is more limited compared to Cocotb,
supporting only Verilator, Icarus Verilog and Synopsys VCS when using Verilog designs. Chiseltest is integrated with
the scalatest testing framework, which allows for an efficient way to organize and run tests.

Chiseltest provides a simple peek/poke/step interface to the simulation and allows for multi-threaded testbenches
using a fork-join model. Work has also been done on formal verification features. However, it doesn't provide any
level of abstraction and infrastructure for building complex
testbenches, focussing on directed unit-test-style testbenches.

\subsubsection{ChiselVerify} %-------------------------------------------------------------------------------------------------

ChiselVerify is a verification library for Chisel designs built on top of Chiseltest \cite{chiselverify}. It aims to improve the verification capabilities in the Chisel ecosystem by providing verification tools such as functional coverage, constrained randomization and formal verification features. While showcasing how a BFM can be built using scala and Chiseltest through a general AXI4 BFM, the framework does not provide guidelines and abstractions on how to build standardized, scalable and reusable testbenches.

Functional coverage can only be collected at the IO of the DUT in ChiselVerify. It is not possible to collect coverage at the transaction level which would require the monitoring of scala variables in addition to Chisel IO ports. However, ChiselVerify allows for the expression of more complex cross coverage relationships. It not only supports simultaneous hits between the cross-coverage bins but also timed relationships, for instance that a hit to a bin results eventually in a hit to the crossed bin.

\begin{comment}

- functional coverage only on ports
- no sampling events for coverage
- but more complex timing relationship for cross coverage, i.e. not just simultaneous hits

- concerns itself with adding systemverilog verification capabilities
- does not provide methods to build reusable scalable testbenches

\end{comment}

\subsubsection{PyUVM} %--------------------------------------------------------------------------------------------------------

PyUVM is a project which aims to make UVM more accessible through the use of Python and by relying on open-source tools \cite{pyuvm}. The project implements a subset of the UVM standard, which they describe as most frequently used, in Python. As a motivation for implementing the project in Python, its ease of use due to light syntax and dynamic typing are highlighted, and the existing verification infrastructure surrounding cocotb. The project tries to take advantage of Pythons flexibility where possible to reduce the verbosity inherent in SystemVerilog. Among the UVM features implemented in PyUVM are the factory, the configuration database, all component types, the TLM interface, UVM phases, sequences and the register abstraction layer.

\subsubsection{UVM support in Verilator} %-------------------------------------------------------------------------------------

In addition to bringing modern verification features to new languages and environments to increase the accessibility, work is also in progress to bring UVM support to Verilator \cite{uvm_verilator}. Verilator is a popular open-source Verilog simulator which transpiles Verilog code to executable C++ models of the hardware design. The effort lead by the Tools Workgroup in CHIPS Alliance aims to make Verilator fully capable of running UVM testbenches. In its current state Verilator only supports a subset of SystemVerilog features. One crucial element is the addition of dynamic scheduling in Verilator which is necessary for dynamically triggered events used in UVM. The effort of enabling all SystemVerilog features required for a full UVM testbench is still in an early phase.

\section{Software Design Patterns} %===========================================================================================


Software design patterns offer general solutions to reoccurring problems in software design. They provide template-like solutions which prescribe a certain structure and organization of code. They represent a variety of best practice approaches to software design. Design patterns can be grouped into three categories. Creational patterns are concerned with making the instantiation of classes more flexible. Structural patterns are concerned with the composition of classes and objects. Lastly, behavioral patterns are concerned with how objects communicate and interact \cite[Ch. 1]{design_patterns}.

One creational design pattern was already encountered when discussing the UVM: the factory pattern. The factory pattern decouples the decision of which object to create from the actual creation of the object. In place of the constructor, a function is used which returns an object with a known interface. The function can be provided in two ways. It could be provided by a so-called abstract factory, which is an object with an interface known to provide the factory function. It could also be provided by a factory method, which derived classes implement to define which object should be created \cite[Ch. 3]{design_patterns}. 

A class C can be written purely relying on the factory for creating objects with interface L. The end-user has the option to define what object with interface L should be created everywhere \textit{without} modifying the code for class C. The factory does not have to be a global singleton as in the UVM, but could also be a parameter to the constructor of class C. Furthermore, a factory is usually specific to a certain group of classes which share an interface and common parameters. This is in contrast to the UVM, where one factory is used for all components, with the only shared parameter being the parent component and the name of the component \cite[Ch. 3]{design_patterns}.

There are other creational design patterns which also facilitate the swapping out of different implementations of interfaces used in a class. The dependency injection pattern relies on the user to actually construct the object it depends upon and then provide them to the class through its constructor or setter-methods. The class works with the object handel it has been provided, without knowing which concrete implementation of the interface it actually works with. Unlike the factory pattern, no infrastructure for object creation is created, but instead the responsibility of object creation is pushed up to the user of the class \cite{ioc_di}.

Another useful pattern is the observer pattern. It is a behavioral pattern, which is also known as the publish-subscribe pattern. In a system where multiple objects, the observers, are interested in changes of the state of another object, the subject, this pattern describes how observers should be notified. It is the subject which maintains a list of observers and has the responsibility to notify them when its state changes \cite[Ch. 5]{design_patterns}. This pattern can be seen in the UVM where analysis components like the scoreboard or coverage collectors are notified of transactions on the DUT interfaces through the monitor. In this case it is not directly a change of state which the observers are notified about, but the occurrence of an event.

\begin{comment}



- factory pattern
- inversion of control \& dependency injection \cite{ioc_di}
- service locator pattern
  - kind of like factory, an object know how to create the right other objects, if global singleton it its kind of like factory
- strategy pattern
- observer pattern


- publish subscribe pattern
- actor pattern \cite{actors}

\end{comment}


\begin{comment}

\section{Software Testing Methods} %===========================================================================================

\subsection{Unit Testing} %----------------------------------------------------------------------------------------------------

\subsection{Integration Testing} %---------------------------------------------------------------------------------------------

\subsection{Property-Based Testing} %------------------------------------------------------------------------------------------

\subsection{Black/Grey/White-Box Testing} %------------------------------------------------------------------------------------

\subsection{Fuzzing} %---------------------------------------------------------------------------------------------------------

\subsection{Mocking} %---------------------------------------------------------------------------------------------------------

\end{comment}

\chapter{Problem Statement \& Methods} %///////////////////////////////////////////////////////////////////////////////////////

The UVM provides abstractions to compartmentalize a testbench in a standardized way. Instead of each engineer or company defining their own abstractions, the UVM provides a shared abstraction. This makes it possible to quickly understand an unknown testbench and to reuse parts of it in other projects. It also makes the exchange of verification components possible and eases the hiring of new verification engineers. 

However, the UVM is a complex and large framework. A lot of the literature on UVM focusses on how the UVM wants us to build testbenches, but not necessarily why it was designed that way. It is clear that the UVM is something which grew out of series of other methodologies and carries a lot of legacy with it. It seems the UVM always tries to anticipate every possible use-case, leading to its size and complexity. It is therefore worth an investigation whether all of its features are actually used in the industry and whether some of the features could be simplified.

While other open-source projects have explored bringing modern verification features like functional coverage, constrained randomization and complex assertions to new languages, little concern has been given to how scalable and flexible verification environments could and should be developed in the respective languages. Languages like Python or Scala used in PyUVM or ChiselVerify bring more features to the table than SystemVerilog in terms of general-purpose programming, but it is not clear how these can be taken advantage of to build scalable and reusable testbenches. Could a productivity gain be achieved by adapting the concepts and abstractions of the UVM and their implementation such that they better fit these high-level languages? Which parts of the UVM should be carried over to such a revised and adapted methodology?



\begin{enumerate}
  \item Is everything in the UVM standard actually used by companies in the industry?
  \item Can the concepts and abstractions of the UVM and their implementation be condensed and simplified?
  \item 
\end{enumerate}

\todo{method}

\begin{comment}

- a lot of the literature talks about HOW the UVM does things, but not necessarily WHY
- I want to try to answer that

- UVM provides abstractions to compartmentalize a testbench
- the engineer would else have to make them up himself -> each engineer would invent their own methodology
- UVM provides a standard way of doing things
- makes it easy to understand an unknown testbench
-> communication is also a big factor of why a methodology should be adapted

- can concepts in UVM be condensed/simplified?
- is all of UVM used/necessary for modern verification?
- is there a productivity gain in embedding a HVL as a DSL in a modern general-purpose language?
- what featues should such a model HVL DSL have?

- UVM seems to attempt to anticipate all possible use cases, but does not achieve this
- shouldn't a framework facilitate the most common use cases and offer a way of extending it for the less common ones?

method:
- informal interview with relevant companies in the industry from the copenhagen area
- look at other modern verification frameworks
- other published sources critically analyzing UVM

\end{comment}

% talk about company interviews
\chapter{Industry Survey} %////////////////////////////////////////////////////////////////////////////////////////////////////

\section{Interview Outline} %==================================================================================================

\section{Company 1} %==========================================================================================================

Company 1 develops ASICs for the hearing aid industry, with a team of 7-8 dedicated verification engineers. The
verification tasks they are performing do not need to adhere to ISO standards directly, but these standards are
already captured in the verification plan. The company prefers using a single language for design to enable
incremental compilation and simplify the workflow. All verification IP (VIP) is developed in-house and actively
maintained, with reuse being a key focus across projects. Examples of reusable VIP include standard interfaces like
APB and SWD. Additionally, they integrate models for external IP, such as EEPROM, using C or Verilog models. Reuse is
made easy and encouraged by maintaining a single code base for everything.

Once the specification for a new project is complete, verification begins in parallel with the design process. The
basic layout of the testbenches can be created early on, based on the interfaces being used. The company uses
module-level testbenches before moving to verification of the top-level. They make extensive use of different UVM
runtime phases, particularly reset phases, to improve modularity and composition. Over time, they have developed
their own standard scoreboard implementation and rely on the configDB to pass data between agents. Connectivity
checks are performed to ensure that specific configurations correctly link different parts of the design.

Their coverage strategy prioritizes functional coverage and FSM code coverage. Debugging is supported by
assertion-based verification, with assertions used for runtime checking, formal checking, or both. Debugging remains
an unpredictable part of the development process. Known bugs are particularly difficult to work around while trying to progress.

For simulation, the company uses VCS and employs continuous integration (CI) to manage regression testing. Each
regression run includes approximately 6000 simulations, with a total runtime of around six hours. Synthesis is also
performed as part of the regression flow, and coverage checks are automatically executed. The CI system ensures that
a freshly checked-out version of the project always works. Although they have considered PyUVM as a potential tool,
they find the speed insufficient for their needs. Furthermore, they are cautious about adopting new or niche tools,
fearing that it could complicate future recruitment efforts.

The testbenches of the company rely heavily on CRV. However, signal data streams are not purely random, but instead
they use typical data encountered in real-world scenarios, because change throughout time is difficult to capture in
constraints. To support DSP verification, MATLAB models are integrated into the verification environment.

The company is overall satisfied with their usage of UVM, but pointed out some shortcomings. Specifically, they
believe that code using the UVM factory is difficult to debug and maintain, while the register abstraction layer
(RAL) in UVM is incomplete. For instance, a single register block cannot be mapped into multiple address spaces, so
they developed their own extension to address this limitation. Concerning the UVM factory, they not that using
callbacks, a feature not available during the development of UVM, would be the better option to keep dependencies
modular in the code. They also observe that many UVM examples available online are outdated and focus only on trivial
cases. Additionally, they note that UVM was influenced by several companies of which some insisted on incorporating
features from their own verification methodologies, sometimes resulting in unnecessary complexity. While they
continue to use UVM, they would are not opposed to a framework with fewer options and simpler design choices. They
also emphasize that power-aware verification remains a significant challenge.

In terms of recruitment, the company intentionally avoids using all UVM and SystemVerilog features to make hiring
easier. They believe that limiting the use of overly complex or non-standard features lowers the learning curve for
new engineers. They have observed that it is rare for software engineers to transition into verification roles, so
they do not see a need to tailor the verification environment specifically for software engineers. To further ease
the process of setting up a verification environment, they use a setup framework that can automatically generate the
basic structure for testbenches and unit tests.

One of the key challenges they face is that the test plan often becomes a bottleneck, more so than the actual
development time for testbenches. While development time tends to be predictable, debugging remains highly
unpredictable, especially when working around known bugs while trying to maintain progress.

\begin{comment}
Company 1 develops ASICs for the hearing aid industry.
- 7-8 dedicated verification engineers

- don't use all UVM and SV features
- try to stick to standard features to make hiring easier
- develop all VIP inhouse
- reuse inhouse VIP across projects, e.g. interfaces like APB or SWD
- integrate C or verilog models for bought IP like EEPROM
- use VCS for simulation
- prefer single language for design to enable incremental compilation
- think that register abstraction in UVM is very complex
- have testbenches at the module level
- use randomized inputs (CRV)
- integrate matlab models for DSP
- Signal data streams are not random but represent typical data
- constraints in time, i.e. between signal packages, are difficult
- use FSM code coverage
- no line or branch coverage
- functional coverage

- after specification is finished, verification starts in parallel with design
- basic testbench layout can be created based on interfaces

- for debugging, assertion-based verification is used, some formal, some runtime, some both
- use connectivity checks to check that certain config connects certain parts of design

- for regression CI is used
- checkout always works
- 6000 simulations with tests
- 6 hours runtime
- synthesis is also done for regression
- coverage checks are performed

- no ISO standards have to be considered by the verification team
- standards are required for chip set, already captured in verification plan

- inhouse VIP is actively improved and maintained and reused
- 1 code base for everything

- have looked at PyUVM but speed is an issue
- want a setup framework to generate basic structure of testbench
- want a setup framework to generate basic unit test structure
- afraid of adapting new niche tools since it may be a recruitment issue

- in their experience it is rare that software engineers become verification engineers, don't see need to tailor verification environment to software engineers

- prefer formal methods over unit tests

- in terms of bottlenecks, the testplan is more of an issue than the actual development time of the testbenches
- Development time is also more predicatable, debugging is not
- especially working around known bugs to further progress is difficult

- use the different UVM runtime phases extensively, especially reset phases, to increase composition
- UVM factory is difficult to debug, and hard to maintain
- callbacks would be the bettern option but weren't available when UVM was drafted

- have their own standard scoreboard implementation

- UVM is not perfect, some companies insisted on things from their own methodology

- use configDB to pass data to agents

- RAL is not finished in their opinion
- e.g. one reg block can't be mapped into multiple address spaces
- developed their own extension

- UVM examples on the internet are often outdated
- only showcase small and trivial examples

- power aware verification is still difficult

- like idea of a framework without too many choices
\end{comment}

\section{Company 2} %==========================================================================================================

Company 2 offers consulting services for hardware design and verification, including training for UVM, thus bringing
extensive experience in handling various verification challenges.

Verification at its core is about building confidence in the design in their opinion. The process typically follows
an evolutionary path: initially, few bugs are found, then many bugs surface as the verification infrastructure
matures, and finally, the design stabilizes with no remaining bugs. This approach often involves dividing the
verification task into smaller, manageable features and verifying them separately. Effective verification requires
robust coverage models, as weak models give misleading results. These coverage models can be validated through
intentionally weak tests, which are expected to score low. However, it is important to note that coverage only
accounts for known-knowns and known-unknowns; there is no established methodology for addressing unknown-unknowns,
i.e. bugs that are not detected by the verification plan. The company follows the best practice of always capturing
design assumptions in assertions.

Formal verification is considered a superior tool for proving design correctness but often struggles with the
complexity of large designs. Despite these challenges, formal methods remain a critical component of their verification process.

They perceive UVM to be a well-working tool for verification. Like all tools, it is not perfect though, and they note
some small issues which they have encountered. The ConfigDB mechanism, can confuse users, in their experience, due to
component scoping mechanism. Misunderstanding its usage can lead to spaghetti code, but when used correctly, the
ConfigDB solves the essential problem of enabling communication between classes without direct references. It acts as
a middleman, allowing information to be passed down from the test to the environment, agents, and drivers. Sometimes
they even use the ConfigDB as a channel, where a driver monitors specific keys for changes to receive data.

They note that the UVM factory mechanism generally works well for flexible component creation. In object-oriented
programming with static hierarchies, the creation of many small specific factories for component substitution in the
hierarchy would be necessary anyways. UVM's global factory addresses this need, reducing the programming overhead for
the end-user. But the UVM factory also introduces overhead, due to the indirection of object creation. Best practices
include using direct instantiation when no overrides are expected or limiting factory use to scenarios where dynamic
changes are really necessary.

Phasing in UVM works well in for the company. But there are some issues concerning the communication of phase
completion. Each component can stop the phase from completing by raising an objection. However, this can lead to
problems when the main test code finishes injecting stimuli while the testbench still has to wait for responses. The
VMM had a consensus object that allowed components to register their “done” conditions, offering a potential
solution. In agents, reset should be handled within the run phase. Base tests should handle reset and shutdown
phases, while derived tests should handle configuration and main phases. The company has developed its own standard
scoreboard infrastructure, which focuses on comparing streams of transactions.

They note that the Register Abstraction Layer (RAL) poses several challenges. It assumes a direct mapping of one
register transaction to one bus sequence item. This assumption can be problematic when protocols require multiple
transactions to complete the changes implied by the register transaction. Additionally, while the RAL supports
writing individual fields, certain protocols do not support this and would need a read-modify-write cycle, resulting
again in multiple sequence items. Register locking is available but does not ensure immutability, raising questions
about its utility. Register randomization can be achieved using the `rand` keyword on fields, but unlike read/write
operations, calls to `rand` are not synchronized if multiple components access the register model in parallel. There
are currently multiple ways of injecting constraints into the register model, but none of them work well or are
verbose according to the company.

Certain types of DUTs present unique challenges. In a scenario where a DUT processes a stream, the interface where
the input stream is consumed may be separated from the one where the output stream is produced. This means the
resulting processed stream items flow though a different agent than the input stream items. An issue arises, if the
test case would like to make a decision based on the processed stream. The test case can only receive feedback from a
driver that it has sent a transaction to. In this case, it would like to receive feedback from a completely different
agent. This case is not considered in UVM. A solution was outlined, where the agent receiving the processed stream
would receive dummy transactions from the test case, such that it could answer with items from the processed stream.

DUTs that only produce outputs, such as random number generators, pose another issue: determining when to stop the
test. Timing-related challenges also arise when one has to determined whether transactions occur simultaneously
across multiple interfaces. The analysis ports of all interfaces would have to send either data or null each clock
cycle, reducing significant overhead, if this was to be determined. Another issue revolves around DUTs like filters,
which require initial stabilization before producing valid output. This could be handled by using metadata in
transactions or RTL signals to indicate readiness.

The company is satisfied with commercial verification tools, but they note that access to these tools remains a
challenge in educational settings, where they are typically available only during courses. They believe that this
educational space should be filled by open-source alternatives such as PyUVM, which they are actively exploring.

\begin{comment}

Company 2 offers consulting services for hardware design and verification, including training for UVM.

- one issue with the RAL is register randomization, they have their own implementation
- scoreboard: have their own standard scoreboard implementation
- model is not in scoreboard, scoreboard only compares streams of transactions

- first step is usually systemC model which can also be used for firmware development
- verification is about model checking
- assertions are alos a model
- verification process is about building confidence, first no bugs, then a lot, then no bugs again
- as verification infrastructure improves, the number of bugs found increases

- UVM falls apart in multi-clock designs

- assertions should be close to the interfaces

- a VIP should already contain a coverage model
- for system level coverage, cross coverage between different agents can be used

- coverage model has to be good, otherwise it is not useful
- can be check with intentionally weak tests

- design assumptions should be captured in assertions

- assertions have to be back-annotated to specification

- verification engineer also has implicit assumptions about how to attack the problem of proving that something is correct

- coverage only contains the known-knowns and known-unknowns
- no methodology to find unknown-unknowns

- for companies, the commercial tools are fine
- they see problem for education, where access to the commercial tools is only available during the course

- verification is about confidence in the design
- divide and conquer, by splitting into features and verifying them separately

- configDB often confuses people due to the scoping mechanism
- can lead to spaghetti code, if one does not know what they are doing
- but it solves one essential problem: talking to a class you don't know
- allows communication between two classes without them holding a handle to each other
- configDB is the middle man
- maybe it would be a good idea to have a configDB without scoping, too
- config is passed down from test -> env -> agents -> driver
- sometimes use configDB as channel -> driver monitors configDB for changes at a certain key

- satisfied with UVM factory
- in OOP program a static hierarchy exists
- if you want to change out components in the hierarchy, you would build your own micro-factory anyways
- why not have a global real factory
- factory comes with overhead though, so use new when it is known that no override will occur
- or limit construction to those times strictly necessary, e.g. at an actual value change

- phasing is a good idea
- reset and shutdown should be provided by base test
- config and main phase should be provided by the derived tests
- agents or any other component should only have a run phase
- one issue is the communication of when a phase is done
- raise and drop objection is used
- but sometimes the test code is done injection stimulus, but the TB should wait until all responses have been received
- VMM had a consensus object where objects register a "done" condition

- in an agent, the reset should also be handled in the run phase

- the scoreboard is application specific, but the infrastructure can be generalized
- this was presented in a paper at DVCon

- one issue is reactive slave
- DUT with interface to producer and consumer, consumer agent should be passive
- but what if test case needs access to values received by consumer?
- need to make consumer active and send one seq item per value, to get feedback
- can't use objections here, else you get stuck

- another issue are DUTs which only have outputs, like an random number generator
- how to know when to stop the test?

- another issue is timing related: when do transactions actually happen at the same time?
- for instance a coverage collector listening to 3 interfaces and should only sample if transaction at the same time on all three interfaces
- in current UVM, monitors would have to send null or transactions every cycle

- another issue are DUTs like filters where some data is needed for the DUT to stabilize and produce valid output
- could use meta data in the transaction
- could also use RTL signal as event to signal that the DUT is ready

- RAL is very big, 25\% of UVM class description
- can write fields individually, what if protocol does not support that? Would need read-modify-write cycle on the bus
- assumes mapping of one register transaction to one bus sequence item, what if multiple are needed?
- should allow translation of register transaction to sequence of bus transactions
- registers have a lock method, but that doesn't mean they are immutable, why is this there?

- randomization of registers can be done by using rand keyword on fields
- calls to rand are not synchronized though, in contrast to read and writes to the register model
- difficult to inject constraints into register model, without being verbose
- also difficult to randomize only one specific reg -> DVCon paper

- general perspective
- verification progress isn't linear -> sometimes hard to communciate between engineers and managers
- the spec isn't always perfect
- sees hope in AI DSL for spec
- formal is the better tool, but struggles with complexity

\end{comment}

\section{Company 3} %==========================================================================================================

Company 3 develops ASICs for the hearing aid industry. Their verification team consists of two dedicated engineers,
though verification responsibilities are shared with design engineers. Design engineers handle simpler verification
tasks, while the verification engineers focus on more complex aspects. The company follows a waterfall development model.

About six years ago, they transitioned to UVM and have since adopted a standardized subset of UVM to ensure
consistency across projects. Agents are frequently reused since most projects rely on the same interfaces. Their
Testbenches are employed not only for functional verification but also for gate-level simulations. While external
VIPs have been used and integrated successfully, most VIPs are developed in-house. Formal methods are used mostly for
their to application-specific instruction-set processors (ASIPs), but outside of that, formal verification is used
only in cases where it is easily applicable.

Their verification framework includes both module-level and top-level testbenches, with UVM environments from the
former being reused in the latter. They also maintain a basic test framework for simpler verification needs. The
employed reference model are mostly transaction-accurate and Matlab models are used for verifying DSP blocks.

The company uses code coverage, but mainly focusses on functional coverage. Like in Company 1, CRV is used except for
the modelling of audio streams where representative data streams are used instead. They note that debugging is one of
the most time-consuming parts of the verification process, often taking up to 70\% of the overall verification effort.

The team makes use of all UVM run phases, setting up the DUT and handling all analysis during the appropriate reset,
configuration and shutdown phases to simplify the work of test case developers. They find the UVM Config DB effective
for parameter passing, in part because they view parameterization of classes in SystemVerilog as cumbersome. Over the
years, they have developed their own UVM extensions, including a custom scoreboard implementation. One issue they
note concerning phasing is the handling of reset during the test. A solution is to do phase-hopping to go back to the
reset phase, which is a messy solution according to them.

Rather than seeing limitations inherent in the UVM, they see more limitations in SystemVerilog itself. For instance
parameterized interfaces can cause issues, and the language's verbosity and boilerplate code could be improved
according to them. Clocking blocks are also avoided due to frequent problems with multiple driver conflicts. While
they generally have no issues with the UVM factory, they believe it would be helpful if unsuccessful overrides
specified via the command line generated explicit error messages. The UVM factory is primarily used by the team to
adjust constraints and replace general sequences with more specific ones when necessary.

Overall, the team places a high priority on maintaining clear and understandable verification code, particularly so
that RTL engineers can easily work with it, too. They prefer straightforward, maintainable code over overly complex
features such as event callbacks.

\begin{comment}

Company 3 develops ASICs for the hearing aid industry.

- 2 dedicated verification engineers

- work in waterfall model
- verification engineer only involved in complex verification tasks, else design engineer does verification

- ISO standard requirements are already encoded in verification plan

- switched to UVM around 6 years ago

- many interfaces are repeated, agents can be reused

- reuse testbenches also for gate level simulation

- have used external VIP, was easy to integrate
- but many develope their own VIP

- there is a "standard" UVM subset

- CRV not used for data audio streams
- coverage is important though

- use code coverage, but disable it for some parts of the design

- use formal methods especially for their ASIPs
- else use formal only in obvious cases

- have their own basic test framework
- module level + top level testbenches
- UVM environment from module level testbenches is reused in top level testbenches

- reference models are cycle or transaction accurate
- some matlab models generate RTL -> model comes for free

- bottleneck is verification, around 70/%
- but out of this, debugging takes the longest time
- soemtimes need to argue why error is actually not an error
- communication overhead and manpower

- use all UVM run phases
- set up everything behind the scenes for the test developer

- config DB works well for them for passing parameters
- thinks that parameterization of classes does not work well in SystemVerilog

- clocking blocks offen suffer from multiple driver problems, because the tools can't understand them
- tend to not use clocking blocks

- does not see a problem with the UVM factory
- only unsuccesful overrides through CLI don't give errors, but should
- are used to change constraints
- used to replace general sequences with specific ones

- have own UVM extensions and methodology
- have own scoreboard implementation

- one issue is reset handling in CRV
- own agent for reset handling
- phase hopping is difficult and messy

- see most issues with SV limitations
- parameterized interfaces
- verbose
- boiler plate

- but prefers guidelines over limitations for UVM

- prefers to keep verification code simle and understandable, also for RTL engineers
- does not like overly complex features like event callbacks

\end{comment}

\section{Takeaways from the Industry Survey} %=================================================================================

- companies are generally satisfied with UVM
- they see it as a tool which may have some shortcomings but it is a standard which is supported by all 3 major EDA vendors
- they only use a subset which seems to be generally accepted across the industry
- where UVM does not satisfy their needs, or does not specify a specific way of handling something (like scoreboard),
they have developed their own extensions

- the approach of having IP units per interface seems to work well, and encourages reuse

- the phasing system seems to be good to facilitate composability of test scenarios

- although formal methods are acknowlegded as superior, where they work, they can't be applied to verify all parts of
a design and a simulation based framework is still needed

- There are different opinions on the UVM factory, it could be interesting to investigate what other options exists
and how user friendly they are

- all agreed that the RAL has some issues
- especially the mapping of one register transaction to one bus sequence item is problematic and easily addressed

- some point out that SystemVerilog itself limits what UVM can do, so it could be interesting to investigate what
other languages could do in the context of hardware verification
- performance is of concern though

- there seem to be some issues related to how the test case code gets feedback from the testbench in order to write
interactive test case code

\chapter{Analysis} %///////////////////////////////////////////////////////////////////////////////////////////////////////////

- here, the requirements should be made and possible approaches outlined
- this includes possible tools to rely on (also for features which are not part of the final product, e.g. coverage and crv)


- there are two levels to the design:
  - the concepts, abstractions that the testbench is strucured by
  - the coding principles which facilitate the flexible nature of the testbench

\section{Scope} %==============================================================================================================

\section{Requirements} %=======================================================================================================

\subsection{Language Support} %------------------------------------------------------------------------------------------------
The first consideration to be made concerns the hardware description languages to be supported by \name.

\subsection{Simulation Backend} %----------------------------------------------------------------------------------------------
The second consideration which has to be made is the simulation backend which the verification framework will use.
This decision is also influenced by the choice of supported HDL's, ease of interfacing with the simulation and
finally of course performance. Since the project itself aims to be open-source, the choice of a proprietary simulator
is not an option. The two most popular open-source simulators are Icarus Verilog and Verilator. Icarus Verilog
compiles a Verilog design into its own format which can be executed by a separate simulation runtime engine, a kind
of interpreter \cite{iverilog}. The newest Verilog standard partially supported is IEEE1800-2012 which includes some
SystemVerilog features.

Verilator, on the other hand, transpiles the Verilog design into C++ or SystemC classes which can then be compiled
into a standalone executable \cite{verilator}. This makes Verilator very suitable for co-simulation with other models
or software components, since it can easily be interfaced at the C/C++ level. The only interface to a verilated
model, apart from setting inputs and reading outputs, is the \texttt{eval} function which runs a static schedule to
update the model's state. Since the model is in the end compiled to a native executable with no additional simulation
runtime, the performance of Verilator is significantly better than Icarus Verilog, with speedups of 100x on a single
thread being reported by the developers \cite{verilator}. Verilator supports nearly fully the IEEE 1364-2005
standard, partially the IEEE 1800-2005 standard and some very specific features of newer standards. The IEEE
1800-2005 standard includes SystemVerilog features, of which some like classes and interfaces are supported by Verilator.

Before version 5, Verilator ignored delay statements in the Verilog source code. The newer version now support this
feature, thus matching the capabilities of Icarus Verilog.

Considering especially the ease of integrating verilated models with other software components in addition to the
superior performance, Verilator was identified as better option for the verification framework.

\subsection{Simulation Interface} %--------------------------------------------------------------------------------------------

- black-box or also white-box/grey-box? we could expose the internal state of the DUT to the testbench (i.e. register access)

\subsection{Concurrency} %-----------------------------------------------------------------------------------------------------

- should allow for event based blocks which fire everytime a certain event occurs -> always blocks
- can be done by checking at start of time step all conditions and adding release events for threads which should be triggered

\subsection{Verification language} %-------------------------------------------------------------------------------------------
The next consideration is

\subsubsection{Dedicated HVL}

\subsubsection{DSL Approach}

\subsection{Coverage} %--------------------------------------------------------------------------------------------------------

\begin{lstlisting}
val g = covergroup {
  coverpoint(a) {
    bin("a0", 0 until 10)
    bin("a1", 10 until 20)
  }
  coverpoint(b) {
    bin("b0", 0 until 10)
    bin("b1", 10 until 20)
  }
}
\end{lstlisting}

functional coverage should be able to collect coverage for scala values! transactions are where coverage is measured

a covergroup with its sampling event should be passed to verification runtime

\subsection{Constrained-Random Stimuli Generation} %---------------------------------------------------------------------------

- for signal processing purposes it may be interesting to have a continuous random signal i.e. a random walk

\subsection{Reusable Verification Components} %--------------------------------------------------------------------------------

- do we actually need static components? why is the driver not just a function? transactors are just functions
mapping one abstraction level to another...
- the sequencer is just a generator, a thing we can get new values from, a driver is just a function, it is the test
case that should call the driver and let that code block until the transaction is done, or fork the interaction to continue

- the whole control flow surrounding sequences in the UVM seems weird
- a sequence aka generator should be passed to a sequencer, which should run in its own thread and generate
transactions on the driver port pulling from different incomin sequences aka a channel mux in gears

\subsection{Testbench Configurability} %---------------------------------------------------------------------------------------

UVM requires a recompilation for each testbench \cite{salemi2013uvm}, we don't need that since a jar can have
multiple entry points.

- configDB should not require error handling all the time, throw exception on miss and have try method
- else parameters as class parameters for components
- causes problem with factory, since factory instantiation only accepts name, else varargs but that is not compile time checked

- static hierarchy, and we want to switch out components -> need to anticipate switch
- why not do dependency injection? then the compiler will help us guarantee that the switch works
- this causes problem with UVM

- if the factory is used, we could pass a small config db to the factory
- each component should then specify its parameters such that the factory checks automatically if all paramters are defined

\subsection{Stimulus Sequences} %----------------------------------------------------------------------------------------------

\subsection{Test Phases} %-----------------------------------------------------------------------------------------------------

\subsection{Register Abstraction} %--------------------------------------------------------------------------------------------

\subsection{Logging} %---------------------------------------------------------------------------------------------------------

\subsection{Debugging} %-------------------------------------------------------------------------------------------------------

\subsection{Level of Testing} %------------------------------------------------------------------------------------------------

\section{Possible Approaches} %================================================================================================

1. SystemVerilog library written for Verilator
2. dedicated HVL compiling via CIRCT dialects to SV testbench
3. DSL cosim with simulation backend

\subsection{SystemVerilog Library} %-------------------------------------------------------------------------------------------

\subsection{HVL using CIRCT} %-------------------------------------------------------------------------------------------------

\subsection{DSL Co-Simulation} %-----------------------------------------------------------------------------------------------

\section{Chosen Approach} %====================================================================================================

\chapter{Implementation} %/////////////////////////////////////////////////////////////////////////////////////////////////////

\section{Simulation Runtime} %=================================================================================================

\subsection{Generating the Verilated Model} %----------------------------------------------------------------------------------

\subsection{Interfacing the Verilated Model} %---------------------------------------------------------------------------------

\subsection{Simulation Interface} %--------------------------------------------------------------------------------------------

\subsection{Simulation Threads} %----------------------------------------------------------------------------------------------

\subsection{Simulation Controller} %-------------------------------------------------------------------------------------------

\section{Verification Environment} %===========================================================================================

\subsection{Phases} %----------------------------------------------------------------------------------------------------------

\subsection{Component Hierarchy} %---------------------------------------------------------------------------------------------

\subsection{Inter-Component Communication} %-----------------------------------------------------------------------------------

\subsection{Component Configuration} %-----------------------------------------------------------------------------------------

\subsection{Stimulus \& Sequences} %-------------------------------------------------------------------------------------------

\subsection{Test Cases} %------------------------------------------------------------------------------------------------------

\subsection{Register Abstraction} %--------------------------------------------------------------------------------------------

\chapter{Results} %////////////////////////////////////////////////////////////////////////////////////////////////////////////

\section{Use-Cases} %----------------------------------------------------------------------------------------------------------

\subsection{Use-Case 1: Simple ALU} %------------------------------------------------------------------------------------------

\subsection{Use-Case 2: Memory-Mapped UART} %----------------------------------------------------------------------------------

\subsection{Use-Case 3: APB Verification IP} %---------------------------------------------------------------------------------

\section{Performance} %--------------------------------------------------------------------------------------------------------

\chapter{Discussion} %/////////////////////////////////////////////////////////////////////////////////////////////////////////

- scala 3 was chosen to use newest features and make future proof but that prevents integration with chisel

- annoying to write interface in scala and SV -> simple tool could create scala code from SV

%\section{Future Work} %========================================================================================================

- verilator performance features (multi-threading, ...)

- add support for attached models
- a function receiving the dut and can do whatever it likes, reading outputs, setting inputs to for instance model a
UART/SPI/Memory
- simulation loop could call the function in each evaluation step

- integration with scalatest

\chapter{Conclusion} %/////////////////////////////////////////////////////////////////////////////////////////////////////////

\printbibliography

\end{document}

\section{Hardware Description Languages}

\section{Hardware Verification Languages}

\section{Concurrency Models}
- talk about how HDL concurrency models differ from other
- talk about other concurrency models
- is UVM actor based?

\subsection{Superlog}

\citeauthor{flake2020a} \cite[Sec. 6, pp. 44-49]{flake2020a}
- engineers at Co-Design Automation Inc. saw potential in conciseness and closeness to HW in Verilog
- C was chosen as a source of inspiration for language extensions due to wide spread use in EDA and embedded systems
communitites
- this turned into superlog
- additions included:
- variable size data types (queues, sparse arrays, associative arrays)
- bundled data types with different directions
- enumerations
- references
- C dpi
- interfaces as collection of wires, but also exposing of methods of modules without hierarchical references

-

\section{UVM}

\section{Transaction-Level Modeling}

\section{Software Testing Methods}
- unit testing
- integration testing
- system testing
- black vs. grey vs. white box testing
