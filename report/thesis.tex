\documentclass[11pt]{article}

\usepackage[backend=biber,style=ieee]{biblatex}
\addbibresource{books.bib}
\addbibresource{papers.bib}


\author{Tjark Petersen}
\title{Thesis}


\begin{document}
\maketitle

\section{Introduction}

- verification has a hard time keeping up with increasing design complexity
- verification is a bottleneck in the design process

\section{Hardware Description Languages}

\section{Hardware Verification Languages}

\section{Concurrency Models}
- talk about how HDL concurrency models differ from other
- talk about other concurrency models
- is UVM actor based?

\section{Constrained-Random Verification}

\cite{Mehta2021}

\section{Code Coverage}

\section{Functional Coverage}

\section{Assertion-Based Verification}

\section{Formal Verification}

\section{System Verilog}

\cite[Sec. 6, pp. 43]{flake2020a}
- by the mid 1990s Verilog began to show its limitations 
- discussed replacing HDLs altogether with C++ or java for hardware design
- other option is strengthening Verilog lang


\subsection{Superlog}

\citeauthor{flake2020a} \cite[Sec. 6, pp. 44-49]{flake2020a}
- engineers at Co-Design Automation Inc. saw potential in conciseness and closeness to HW in Verilog
- C was chosen as a source of inspiration for language extensions due to wide spread use in EDA and embedded systems communitites
- this turned into superlog
- additions included:
  - variable size data types (queues, sparse arrays, associative arrays)
  - bundled data types with different directions
  - enumerations
  - references
  - C dpi
  - interfaces as collection of wires, but also exposing of methods of modules without hierarchical references

\subsection{Vera}

\cite[Sec. 7, pp. 51-??]{flake2020a}
- 

\section{UVM}

\section{Event-Driven Simulation}

\section{Cycle-Based Simulation}

\section{Transaction-Level Modeling}

\section{Software Testing Methods}
- unit testing
- integration testing
- system testing
- black vs. grey vs. white box testing


\printbibliography

\end{document}
