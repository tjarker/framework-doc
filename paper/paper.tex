\documentclass[conference]{IEEEtran}
\IEEEoverridecommandlockouts
% The preceding line is only needed to identify funding in the first footnote. If that is unneeded, please comment it out.
\usepackage{cite}
\usepackage{amsmath,amssymb,amsfonts}
\usepackage{algorithmic}
\usepackage{graphicx}
\usepackage{textcomp}
\usepackage{xcolor}
\def\BibTeX{{\rm B\kern-.05em{\sc i\kern-.025em b}\kern-.08em
    T\kern-.1667em\lower.7ex\hbox{E}\kern-.125emX}}
\begin{document}

\title{Conference Paper Title*\\
{\footnotesize \textsuperscript{*}Note: Sub-titles are not captured in Xplore and
should not be used}
\thanks{Identify applicable funding agency here. If none, delete this.}
}

\author{\IEEEauthorblockN{1\textsuperscript{st} Given Name Surname}
\IEEEauthorblockA{\textit{dept. name of organization (of Aff.)} \\
\textit{name of organization (of Aff.)}\\
City, Country \\
email address or ORCID}
\and
\IEEEauthorblockN{2\textsuperscript{nd} Given Name Surname}
\IEEEauthorblockA{\textit{dept. name of organization (of Aff.)} \\
\textit{name of organization (of Aff.)}\\
City, Country \\
email address or ORCID}
\and
\IEEEauthorblockN{3\textsuperscript{rd} Given Name Surname}
\IEEEauthorblockA{\textit{dept. name of organization (of Aff.)} \\
\textit{name of organization (of Aff.)}\\
City, Country \\
email address or ORCID}
}

\maketitle

\begin{abstract}
This document is a model and instructions for \LaTeX.
This and the IEEEtran.cls file define the components of your paper [title, text, heads, etc.]. *CRITICAL: Do Not Use Symbols, Special Characters, Footnotes, 
or Math in Paper Title or Abstract.
\end{abstract}

\begin{IEEEkeywords}
component, formatting, style, styling, insert
\end{IEEEkeywords}

\section{Introduction}

Functional verification is an integral part of the development process of a digital system. It is not only needed to ensure functional correctness but also required to prove the fulfillment of specific safety requirements in safety-critical systems. 
With ever more complex heterogeneous systems-on-chips (SoCs) coming with a large set of functional requirements, the task of verification has gotten a bottleneck in the development process requiring up to 2/3 of the development time \cite{bergeron2012writing}.

Simulation based verification techniques have evolved from simple directed testbenches to more sophisticated approaches using constrained random stimulus alongside monitoring of functional coverage, based on the stimulus as well as assertions. 
The \textit{Universal Verification Methodology} (UVM) is a widely adopted IEEE standard which provides a framework for creating sophisticated testbenches in SystemVerilog that encourages reuse by providing guidelines and templates for common building blocks \cite{flake2020a}. 

While work is in progress of enabling the UVM on Verilator, an open-source (System)Verilog simulator, the UVM in SystemVerilog is for now only fully supported by commercial simulators.
In the open-source community, the \textit{cocotb} project provides a Python-based co-simulation environment, while the \textit{pyUVM} project aims to provide a Python-based UVM-like framework. The \textit{pyVSC} project adds random stimulus generation and coverage collection capabilities. Other projects like \textit{Chisel} or \textit{SpinalHDL} have their own testbench frameworks, which only provide barebones means to control a simulation and drive stimulus. The \textit{ChiselVerify} project added some UVM-like features on top of the \textit{chiseltest} framework like coverage collection and constrained random stimulus generation. With chiseltest getting deprecated, the future of ChiselVerify is uncertain.

Initial talks with two Copenhagen based companies working with verification have shown that the UVM is not perfect. The guidelines are still quite loose, leading to different ways of achieving the same goal. For instance the implementation of a scoreboard isn't standardized. The Register Abstraction Layer (RAL) wasn't received as being complete either, leading to both companies developing internal extensions.

While Python is a language with a low entry barrier, it does not scale well with complex systems such as large testbench environments where features like compile-time type safety can circumvent run-time crashes and provide larger confidence in the correctness of the implemented code. This is especially important when reuse is encouraged such that the internals of verification components aren't always known to the developer. Instead a statically typed language should be used. On the other hand, the verbosity of the language should be minimal while offering garbage collection and a rich standard library to ease the development process. 

This opens the oportunity to reconsider the abstractions behind the UVM as well as their respective implementation in the context of a new open-source framework. This framwork should accept synthesizable SystemVerilog designs to make it as accessible as possible, given that (System)Verilog is the most widely used language for digital design and used as an intermediate language by other hardware description languages like Chisel, MyHDL, Clash or SpinalHDL.

\textit{Scala} is a language that fits the requirements for a compiled, garbage-collected language with a strong type system. At its core, the language is simple and concise but it offers more advanced features like functional programming. The language is furthermore especially friendly to DSLs, which should allow for natural APIs for the user of a verification framework. It it therefore chosen as the language for the implementation of the proposed framework.

\section{Verification}

Simulation based verification at its core is about applying stimulus to a design and checking the observed behavior against some form of a model. At its most primitive form, this can be done by writing a testbench, where generation of the stimulus, driving of the stimulus and checking of the observed behavior is intertwined. 

This works for small designs with one interface and few features. But as the complexity increases, a more modular approach which divides the responsibilities in a testbench between different components is needed. This modularity allows for reuse of components and for the introduction of abstractions. These abstractions form the infrastructure on which a series of tests targeting specific features can be built.

\section*{Acknowledgment}

The preferred spelling of the word ``acknowledgment'' in America is without 
an ``e'' after the ``g''. Avoid the stilted expression ``one of us (R. B. 
G.) thanks $\ldots$''. Instead, try ``R. B. G. thanks$\ldots$''. Put sponsor 
acknowledgments in the unnumbered footnote on the first page.

\section*{References}

Please number citations consecutively within brackets \cite{b1}. The 
sentence punctuation follows the bracket \cite{b2}. Refer simply to the reference 
number, as in \cite{b3}---do not use ``Ref. \cite{b3}'' or ``reference \cite{b3}'' except at 
the beginning of a sentence: ``Reference \cite{b3} was the first $\ldots$''

Number footnotes separately in superscripts. Place the actual footnote at 
the bottom of the column in which it was cited. Do not put footnotes in the 
abstract or reference list. Use letters for table footnotes.

Unless there are six authors or more give all authors' names; do not use 
``et al.''. Papers that have not been published, even if they have been 
submitted for publication, should be cited as ``unpublished'' \cite{b4}. Papers 
that have been accepted for publication should be cited as ``in press'' \cite{b5}. 
Capitalize only the first word in a paper title, except for proper nouns and 
element symbols.

For papers published in translation journals, please give the English 
citation first, followed by the original foreign-language citation \cite{b6}.

\begin{thebibliography}{00}
\bibitem{b1} G. Eason, B. Noble, and I. N. Sneddon, ``On certain integrals of Lipschitz-Hankel type involving products of Bessel functions,'' Phil. Trans. Roy. Soc. London, vol. A247, pp. 529--551, April 1955.
\bibitem{b2} J. Clerk Maxwell, A Treatise on Electricity and Magnetism, 3rd ed., vol. 2. Oxford: Clarendon, 1892, pp.68--73.
\bibitem{b3} I. S. Jacobs and C. P. Bean, ``Fine particles, thin films and exchange anisotropy,'' in Magnetism, vol. III, G. T. Rado and H. Suhl, Eds. New York: Academic, 1963, pp. 271--350.
\bibitem{b4} K. Elissa, ``Title of paper if known,'' unpublished.
\bibitem{b5} R. Nicole, ``Title of paper with only first word capitalized,'' J. Name Stand. Abbrev., in press.
\bibitem{b6} Y. Yorozu, M. Hirano, K. Oka, and Y. Tagawa, ``Electron spectroscopy studies on magneto-optical media and plastic substrate interface,'' IEEE Transl. J. Magn. Japan, vol. 2, pp. 740--741, August 1987 [Digests 9th Annual Conf. Magnetics Japan, p. 301, 1982].
\bibitem{b7} M. Young, The Technical Writer's Handbook. Mill Valley, CA: University Science, 1989.
\end{thebibliography}
\vspace{12pt}
\color{red}
IEEE conference templates contain guidance text for composing and formatting conference papers. Please ensure that all template text is removed from your conference paper prior to submission to the conference. Failure to remove the template text from your paper may result in your paper not being published.

\end{document}
